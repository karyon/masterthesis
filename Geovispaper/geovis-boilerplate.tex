% This is "sig-alternate.tex" V2.1 April 2013
% This file should be compiled with V2.5 of "sig-alternate.cls" May 2012
%
% This example file demonstrates the use of the 'sig-alternate.cls'
% V2.5 LaTeX2e document class file. It is for those submitting
% articles to ACM Conference Proceedings WHO DO NOT WISH TO
% STRICTLY ADHERE TO THE SIGS (PUBS-BOARD-ENDORSED) STYLE.
% The 'sig-alternate.cls' file will produce a similar-looking,
% albeit, 'tighter' paper resulting in, invariably, fewer pages.
%
% ----------------------------------------------------------------------------------------------------------------
% This .tex file (and associated .cls V2.5) produces:
%       1) The Permission Statement
%       2) The Conference (location) Info information
%       3) The Copyright Line with ACM data
%       4) NO page numbers
%
% as against the acm_proc_article-sp.cls file which
% DOES NOT produce 1) thru' 3) above.
%
% Using 'sig-alternate.cls' you have control, however, from within
% the source .tex file, over both the CopyrightYear
% (defaulted to 200X) and the ACM Copyright Data
% (defaulted to X-XXXXX-XX-X/XX/XX).
% e.g.
% \CopyrightYear{2007} will cause 2007 to appear in the copyright line.
% \crdata{0-12345-67-8/90/12} will cause 0-12345-67-8/90/12 to appear in the copyright line.
%
% ---------------------------------------------------------------------------------------------------------------
% This .tex source is an example which *does* use
% the .bib file (from which the .bbl file % is produced).
% REMEMBER HOWEVER: After having produced the .bbl file,
% and prior to final submission, you *NEED* to 'insert'
% your .bbl file into your source .tex file so as to provide
% ONE 'self-contained' source file.
%
% ================= IF YOU HAVE QUESTIONS =======================
% Questions regarding the SIGS styles, SIGS policies and
% procedures, Conferences etc. should be sent to
% Adrienne Griscti (griscti@acm.org)
%
% Technical questions _only_ to
% Gerald Murray (murray@hq.acm.org)
% ===============================================================
%
% For tracking purposes - this is V2.0 - May 2012

\documentclass{acm_proc_article-sp}


\usepackage{epstopdf}
\usepackage{outlines}
\usepackage{cleveref}
\usepackage{tabulary}
\usepackage{pbox}
\usepackage[labelfont=bf,font=bf,skip=3pt]{caption}
% \captionsetup[figure]{labelfont=bf}


\begin{document}

% Copyright
%\setcopyright{acmcopyright}
%\setcopyright{acmlicensed}
%\setcopyright{rightsretained}
%\setcopyright{usgov}
%\setcopyright{usgovmixed}
%\setcopyright{cagov}
%\setcopyright{cagovmixed}




\title{Improved Imperfect Shadow Maps}
%\subtitle{[Extended Abstract]}
%
% You need the command \numberofauthors to handle the 'placement
% and alignment' of the authors beneath the title.
%
% For aesthetic reasons, we recommend 'three authors at a time'
% i.e. three 'name/affiliation blocks' be placed beneath the title.
%
% NOTE: You are NOT restricted in how many 'rows' of
% "name/affiliations" may appear. We just ask that you restrict
% the number of 'columns' to three.
%
% Because of the available 'opening page real-estate'
% we ask you to refrain from putting more than six authors
% (two rows with three columns) beneath the article title.
% More than six makes the first-page appear very cluttered indeed.
%
% Use the \alignauthor commands to handle the names
% and affiliations for an 'aesthetic maximum' of six authors.
% Add names, affiliations, addresses for
% the seventh etc. author(s) as the argument for the
% \additionalauthors command.
% These 'additional authors' will be output/set for you
% without further effort on your part as the last section in
% the body of your article BEFORE References or any Appendices.

\numberofauthors{1} %  in this sample file, there are a *total*
% of EIGHT authors. SIX appear on the 'first-page' (for formatting
% reasons) and the remaining two appear in the \additionalauthors section.
%
\author{
% You can go ahead and credit any number of authors here,
% e.g. one 'row of three' or two rows (consisting of one row of three
% and a second row of one, two or three).
%
% The command \alignauthor (no curly braces needed) should
% precede each author name, affiliation/snail-mail address and
% e-mail address. Additionally, tag each line of
% affiliation/address with \affaddr, and tag the
% e-mail address with \email.
%
% 1st. author
\alignauthor
Johannes Linke\\
       \affaddr{Hasso Plattner Institute}\\
       %\affaddr{1932 Wallamaloo Lane}\\
       %\affaddr{Wallamaloo, New Zealand}\\
       \email{johannes.linke@student.hpi.de}
}

\maketitle


%!TEX root = geovis-boilerplate.tex


\begin{abstract}
TODO
\end{abstract}



\section{Introduction}
For decades, computer graphics researchers have strived to achieve the faithful reproduction of reality in synthetic renderings. While photorealism has been achieved in offline rendering contexts, many optical effects occurring in physical environments are still not in use in interactive and real-time applications or implemented with severe limitations. Several lighting effects collectively referred to as ``global illumination'', such as caustics, subsurface scattering and diffuse and specular indirect light, fall into this category.

In this paper, we develop a system that simulates diffuse indirect light in arbitrary and dynamic scenes using many-light techniques. We will trade in quality and instead focus on performance while maintaining scalability with the goal to achieve real-time performance on commodity hardware.




\section{Related Work}

Many-light methods for simulating global illumination effects are a well-researched topic. They are based on Instant Radiosity \cite{Keller:1997:InstantRadiosity}, in which photon mapping is used to create small lights, called Virtual Point Lights (VPLs) at every intersection of photons with the scene geometry.


Most of the work on many-light methods can be categorized into the areas VPL placement, solving occlusion, gathering light into the framebuffer and mitigating singularities, an artifact that occurs near positions of VPLs.


The foundation of many algorithms for VPL placement are Reflective Shadow Maps (RSMs) \cite{Dachsbacher:2005:RSM}, which are shadow maps with additional surface information. Using the additional data, VPLs are created per texel of the RSM. View-adaptive placement of VPLs greatly enhances performance and quality \cite{ritschel2011ismsViewAdaptive}. \cite{prutkin2012reflective} cluster RSM samples to virtual area lights, and \cite{hedman2016sequential} focuses on temporal coherence when placing VPLs to avoid flickering.

A common approach for solving occlusion are Imperfect Sha\-dow Maps (ISMs) \cite{ritschel2008ism}, which use a point-based approximation of the scene to efficiently render a small, approximate shadow map for hundreds of VPLs simultaneously. ISMs have been used in production for rendering many spotlights with shadows \cite{evans2015dreams}, and several enhancements have been proposed \cite{ritschel2011ismsViewAdaptive, hollander2011manylods, barak2013temporally}. Using slim voxels \cite{sugihara2014layered, sun2015manylightsSVO, chen2016quantizing} for occlusion is another approach that needs only a single bit per voxel for visibility testing, thereby avoiding the usually high memory requirements of voxelization techniques.

\cite{dachsbacher2006splatting, Nichols:2009:splatting} are using splatting techniques in favor of the more common gathering approach. \cite{sloan2007image, laine2007incremental} provide details on how to speed up the straight-forward approach of iterating over sets of VPLs per fragment.

In na\"ive implementations of many-light methods, bright spots will appear near the VPL's positions due to the light's attenuation term approaching infinity. A common approach is to clamp the term. This introduces bias, which can be compensated e.g. in screen space \cite{novak2011screen}. Those singularities can also be avoided through more advanced light representations \cite{tokuyoshi2015vsgl}. \cite{olsson2012clustered}



\section{Visibility Computation with Imperfect Shadow Maps}
\label{sec:concept:ism}

The original paper \cite{ritschel2008ism} converts the scene geometry to a point set in a preprocessing step and uses the points to efficiently render hundreds of shadow maps in parallel. They use splatting to render the points and fill the resulting holes in the shadowmaps with a pull-push algorithm inspired by \cite{Marroquim:2007:reconstruction}.

\cite{ritschel2011ismsViewAdaptive} build on this by converting the scene to a triangle texture dynamically, and sampling the points from that texture. Instead of computing a triangle texture, \cite{barak2013temporally} use the tessellation units of recent GPUs to dynamically convert triangles into points.
We follow this approach since it is relatively simple to implement and inherently dynamic, but have not implemented the adaptive sampling from \cite{ritschel2011ismsViewAdaptive} yet.
\cite{ritschel2008ism} also present multi-bounce indirect illumination with their technique, which we have not implemented.

In contrast to \cite{ritschel2008ism}, we have found the pull-push postprocessing to be of little benefit when using regular point splatting. Instead, we propose to render single-pixel points, which opens interesting optimization opportunities but in turn does in fact require the pull-push algorithm to reconstruct the scene surfaces.

We describe the regular point splat rendering process in the following section and subsequently detail the pull-push algorithm that we use when rendering single-pixel points.


\subsection{Point Rendering with Splatting}

To convert the scene to be rendered into points, all its triangles are first tesselated to meet a certain maximum size in order for the point conversion to not introduce extreme inaccuracies. Of the smaller tesselated triangles, the center and area are calulated and a point with matching center and area is created. While it might be more performant to create points just on the vertices of the triangles, it will at the same time be more inaccurate since this approach will enlarge the rendered area considerably over the triangle's extents.

Now a random VPL is chosen per point and backface culling is performed. Likewise, points that lie behind their VPL's hemisphere are culled before splatting the point into the respective imperfect shadow map. For performance reasons, a simple paraboloid projection is used for the ISMs in lieu of conventional cubemaps. Here, another advantage of using points comes into play: Compared to triangles, it is trivial to perform a paraboloid projection on them.

As an optimization, we iterate over several VPLs per point and collect those that pass the culling test and render to all of them. More on this in \cref{chap:implementation}.

To remove holes in the resulting shadowmap, the original paper implements a simplified variant of \cite{Marroquim:2007:reconstruction} that only uses depth information from the point rendering. We found that this approach has little benefit over simply enlarging all the point splats during rendering, therefore we did not use it when using splat rendering.



\subsection{Point Rendering with Pull-Push Postprocessing}

As an alternative to point splat rendering, this subsection proposes a second approach that follows the algorithm of \cite{Marroquim:2007:reconstruction} more closely with the intent to obtain higher-quality shadowmaps. More specifically, the points are rendered as a single pixel, albeit with additional attributes like size and normal. These are then used in a subsequent reconstruction pass to create an (ideally) hole-free shadowmap. This approach also allows a point to be rendered into multiple shadowmaps with a moderate performance impact, more on this in the next chapter.

To provide a rough overview, the pull-push algorithm proposed by \cite{Marroquim:2007:reconstruction} is a ``pyramid-method'' (see \cite{Strengert:2006:Pyramid}) that uses the mipmap levels of a texture to first condense and then spread the data in the texture, in this case the sparse point set representing the scene. During the pull phase, it aggregates the information from four pixels of a finer level to one pixel of a coarser level. The subsequent push phase aggregates four pixels of a coarser level to one pixel of a finer level, but only if the pixel of the finer level contains no data or is determined to belong to an occluded surface. This way, closed surfaces are derived from single-pixel points.

\subsubsection{Pull Phase}
The pull phase reads four pixels of a finer level, computes an aggregate representation of them and writes the result to one pixels of the next coarser level. However, it only considers pixels that pass the following tests: First, they need to contain a valid depth at all (i.\,e.\ a point must have been rendered into this pixel) and second, the point in this pixel must not be occluded. A point is considered to be occluded if it is outside  the depth interval of the frontmost of the four pixel used for interpolation.

Figuratively speaking, the depth interval of a point denotes the range in which other points are considered to belong to the same surface and can be used for interpolation. For the rendered points, the depth interval is simply $[d;d+k]$ where $d$ is the points depth and $k$ an arbitrarily chosen constant. For interpolated points, the new depth interval spans the interpolated depth of the point to the highest depth value of the depth intervals of all ``parent'' points.

To calculate the attributes for a new point, the depth and radius of the parent points are interpolated in between with equal weights. Since the center of the aggregated point might not match with the position of the pixel (this happens if not all four pixels are used for interpolation), a displacement vector is also calculated to accurately describe the point's location.


\subsubsection{Push Phase}

Analogous to the pull phase, the push phase reads four pixels of the coarser level, rejects pixels that don't pass certain tests, and writes one output pixel into the finer level. First, just like in the pull phase, the input pixels must contain a valid depth and not be occluded to be considered. Additionally, a radius check is performed: If the pixel's location is outside the point's extends described by the displacement vector and its radius, then the point is not considered either. This is done to limit a point's influence to its actual size.

The target output location in the finer level might already contain data computed during the pull phase. This data is overwritten if the aggregated data has a smaller depth, i.e. if the original data is considered to be occluded. Otherwise, the original point is left untouched, as it is potentially more accurate than the just computed, interpolated point.

Since the input and output pixel's location do not match the way they do during the pull phase, the weights used for interpolation during the push phase are not uniform, see Figure~\ref{fig:???}.

 \cite{Marroquim:2007:reconstruction} also use the point's normals to correctly limit the point's size, and \cite{Marroquim:2008:reconstruction2} propose gaussian weights based on the pixel's distance to the point's location, instead constant weights solely based on pixel locations. Both of these additions we have not implemented yet.



%
% The following two commands are all you need in the
% initial runs of your .tex file to
% produce the bibliography for the citations in your paper.
\bibliographystyle{abbrv}
\bibliography{../foo-thesis}  % sigproc.bib is the name of the Bibliography in this case
% You must have a proper ".bib" file
%  and remember to run:
% latex bibtex latex latex
% to resolve all references
%
% ACM needs 'a single self-contained file'!
%
%APPENDICES are optional
%\balancecolumns


%\balancecolumns % GM June 2007
% That's all folks!
\end{document}
